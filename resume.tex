% Copyright 2013 Christophe-Marie Duquesne <chmd@chmd.fr>
% Copyright 2014 Mark Szepieniec <http://github.com/mszep>
% 
% ConText style for making a resume with pandoc. Inspired by moderncv.
% 
% This CSS document is delivered to you under the CC BY-SA 3.0 License.
% https://creativecommons.org/licenses/by-sa/3.0/deed.en_US

\startmode[*mkii]
  \enableregime[utf-8]  
  \setupcolors[state=start]
\stopmode

\setupcolor[hex]
\definecolor[titlegrey][h=757575]
\definecolor[sectioncolor][h=397249]
\definecolor[rulecolor][h=9cb770]

% Enable hyperlinks
\setupinteraction[state=start, color=sectioncolor]

\setuppapersize [A4][A4]
\setuplayout    [width=middle, height=middle,
                 backspace=20mm, cutspace=0mm,
                 topspace=10mm, bottomspace=20mm,
                 header=0mm, footer=0mm]

%\setuppagenumbering[location={footer,center}]

\setupbodyfont[11pt, helvetica]

\setupwhitespace[medium]

\setupblackrules[width=31mm, color=rulecolor]

\setuphead[chapter]      [style=\tfd]
\setuphead[section]      [style=\tfd\bf, color=titlegrey, align=middle]
\setuphead[subsection]   [style=\tfb\bf, color=sectioncolor, align=right,
                          before={\leavevmode\blackrule\hspace}]
\setuphead[subsubsection][style=\bf]

\setuphead[chapter, section, subsection, subsubsection][number=no]

%\setupdescriptions[width=10mm]

\definedescription
  [description]
  [headstyle=bold, style=normal,
   location=hanging, width=18mm, distance=14mm, margin=0cm]

\setupitemize[autointro, packed]    % prevent orphan list intro
\setupitemize[indentnext=no]

\setupfloat[figure][default={here,nonumber}]
\setupfloat[table][default={here,nonumber}]

\setuptables[textwidth=max, HL=none]

\setupthinrules[width=15em] % width of horizontal rules

\setupdelimitedtext
  [blockquote]
  [before={\setupalign[middle]},
   indentnext=no,
  ]


\starttext

\section[fabien-dubois]{Fabien Dubois}

\subsubsection[alternance-master-informatique-ipii]{Alternance Master
Informatique IPII}

\thinrule

\startblockquote
Moi ? 🙋🏻‍♂️ Passionné par l'informatique et les nouvelles technologies,
partisan de l'innovation et de la digitalisation. Je veux
{\bf {\em Apprendre}}, {\bf {\em Découvrir}} et {\bf {\em Évoluer}} avec
vous ! 🙂
\stopblockquote

\thinrule

\subsection[etudes]{Etudes}

\startdescription{2016-2017}
  {\bf Licence 3 Informatique Alternance} : Université de Valenciennes,
  UVHC.

  Licence 3 effectuée en alternance chez VINCI Construction France
\stopdescription

\startdescription{2014-2016}
  {\bf DUT Informatique Alternance} : Université de Valenciennes,
  antenne de Maubeuge.

  DUT Informatique dont la deuxième année a été réalisé en alternance
  chez VINCI Construction France
\stopdescription

\startdescription{2013-2014}
  {\bf Bac Scientifique} : Lycée Louis Pasteur Somain
\stopdescription

\subsection[expériences-professionnelles]{Expériences Professionnelles}

{\bf VINCI Construction France :}

Technicien en alternance au service informatique :

\startitemize
\item
  Mise en place d'une solution de gestion et de supervision du parc
  informatique,
  \useURL[url1][https://www.eyesofnetwork.com/?lang=fr][][Eyes Of
  Network & GLPI]\from[url1]
\item
  Administration systèmes et réseaux
\item
  Support aux utilisateurs
\stopitemize

{\bf Animateur en centre de loisirs}

Also with a short description.

\subsection[expériences-techniques]{Expériences Techniques}

\startdescription{Mise en place d'une solution de gestion et de
supervision de parc informatique}
  {\bf déploiment de la solution} Mise en place de la solution Open
  Source contenant GLPI et Nagios

  {\bf gestion du projet} Gestion du Projet
\stopdescription

\startdescription{Languages de Programmation}
  {\bf first-lang:} Here, we have an itemization, where we only want to
  add descriptions to the first few items, but still want to mention
  some others together at the end. A format that works well here is a
  description list where the first few items have their first word
  emphasized, and the last item contains the final few emphasized terms.
  Notice the reasonably nice page break in the pdf version, which
  wouldn't happen if we generated the pdf via html.

  {\bf second-lang:} Description of your experience with second-lang,
  perhaps again including a
  \useURL[url2][https://github.com/fabiendubois/][][link]\from[url2],
  this time placing the url reference elsewhere in the document to
  reduce clutter (see source file).

  {\bf obscure-but-impressive-lang:} We both know this one's pushing it.

  Basic knowledge of {\bf C}, {\bf x86 assembly}, {\bf forth},
  {\bf Common Lisp}
\stopdescription

\thinrule

\startblockquote
\useURL[url3][mailto:fabiendubois2304@gmail.com][][fabiendubois2304@gmail.com]\from[url3]
\stopblockquote

\stoptext
